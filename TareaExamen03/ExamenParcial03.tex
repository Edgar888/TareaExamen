\documentclass{article}
\usepackage{tikz}
\usetikzlibrary{arrows.meta, positioning}

\usepackage[utf8]{inputenc}
\usepackage[spanish,mexico]{babel}
\usepackage{listings}
\usepackage{amsmath}
\setlength{\textwidth}{18cm}
\setlength{\oddsidemargin}{-1cm}
\setlength{\headsep}{1cm}
\setlength{\voffset}{0cm}
\setlength{\topmargin}{0cm}
\setlength{\headheight}{0cm}
\usepackage{tikz}
\usepackage{semantic}
\usepackage{url}
\usetikzlibrary{positioning}
\usetikzlibrary{calc,arrows}
\usepackage{multicol}
\usepackage{lipsum} 
\usepackage{multirow}


\usepackage{amsmath}

\usepackage{graphicx}
\usepackage{forest}
\usepackage{tikz-qtree}
\usepackage{xcolor}

\begin{document}
\pagecolor{black}
\color{white}

%%%%%% ENCABEZADO %%%%%%%%%%%%%%%%%%%%%%%%%%%%%%%%%%%%%%%
    \colorbox{black}{
        \begin{minipage}[t]{0.16 \textwidth}
           \begin{flushright}
            \includegraphics[width=1in]{UNAM.png}
           \end{flushright}
        \end{minipage}
        \begin{minipage}[H]{0.62 \textwidth}
            \begin{center}
                {\large \textsc{Universdad Nacional Autónoma de México}}
                \vspace{0.25cm}
                \\
                { \large \textbf{Lenguajes de Programacion\\ Examen Parcial III}}                
                \textbf{}
                \begin{multicols}{2}
                \begin{flushleft}
                \begin{itemize}
                    % NOMBRES DE INTEGRANTES
                    \item  \small Edgar Montiel Ledesma\\ 317317794
    
                    \item \footnotesize Carlos Daniel Cortes Jimenez\\ 420004846
                \end{itemize}
                \end{flushleft}
                \vspace{0.25cm}
                \end{multicols} 
            \end{center}
            \vspace{0.05cm}
        \end{minipage}
        \begin{minipage}[t]{0.16 \textwidth}
            \begin{flushleft}
                \includegraphics[width=1in]{EFC.png}
            \end{flushleft}
        \end{minipage}
    }
    
    \begin{tikzpicture}
        \draw[thick] (-6.5,0)--(11.2,0);
    \end{tikzpicture}
    %%%%%%%%%%%%%%%%%%%%%%%%%%%%%%%%%%%%%%%%%%%%%%%%%%%%%%%%%
    \section{Problema}
    El constructor alternativo if se puede generalizar mediante un operador case definido como sigue:

        \begin{itemize}
            \item[ ] e ::= . . . \,| case g end
            \item[ ] g ::= e1 $\Rightarrow$ e2 \,| g ; g 
        \end{itemize}

    Una expresion de la forma e1 $\Rightarrow$ e2 se conoce como expresion resguardada, siendo la expresion e1 una expresion booleana llamada guardia. Un constructor case se evalua como sigue: recorrer las expresiones resguardadas ei $\Rightarrow$ ej en orden de izquierda a derecha (respectivamente de arriba a abajo), hasta hallar la primera expresion resguardada, digamos ek $\Rightarrow$ el tal que ek $\Rightarrow$ $*$ true, en cuyo caso se procede a evaluar el, cuyo valor final es tambien el resultado de la evaluacion de la expresion case. Por ejemplo considerese el siguiente programa:

        \begin{itemize}
            \item[ ] case x=0 => x ;
            \item[ ] x>2 => xˆ2 ;
            \item[ ] x>0 => x-2 ;
            \item[ ] x<0 => x+2
            \item[ ] end
        \end{itemize}

    \section{Preguntas}
    \begin{itemize}
        \item[1.] Define la sintaxis abstracta del operador case.
        \item[2.] Define las reglas de transicion para modelar la semantica operacional del nuevo operador case.
        \item[3.] Define las reglas de tipado para la semantica estatica del operador case.
        \item[4.] Extiende el algoritmo de inferencia de tipos de la nota 8, agregando las reglas de generacion de restricciones para el operador case.
        \item[5.] Explica por que el operador case es azucar sintactica en el lenguaje.
    \end{itemize}

\end{document}
