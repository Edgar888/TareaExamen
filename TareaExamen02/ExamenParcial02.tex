\documentclass{article}
\usepackage{tikz}
\usetikzlibrary{arrows.meta, positioning}

\usepackage[utf8]{inputenc}
\usepackage[spanish,mexico]{babel}
\usepackage{listings}
\usepackage{amsmath}
\setlength{\textwidth}{18cm}
\setlength{\oddsidemargin}{-1cm}
\setlength{\headsep}{-1cm}
\setlength{\voffset}{0cm}
\setlength{\topmargin}{0cm}
\setlength{\headheight}{0cm}
\usepackage{tikz}
\usetikzlibrary{positioning}
\usetikzlibrary{calc,arrows}
\usepackage{multicol}
\usepackage{lipsum} 
\usepackage{multirow}


\usepackage{amsmath}

\usepackage{graphicx}
\usepackage{forest}
\usepackage{tikz-qtree}
\usepackage{xcolor}

\begin{document}
\pagecolor{black}
\color{white}

%%%%%% ENCABEZADO %%%%%%%%%%%%%%%%%%%%%%%%%%%%%%%%%%%%%%%
    \colorbox{black}{
        \begin{minipage}[t]{0.16 \textwidth}
           \begin{flushright}
            \includegraphics[width=1in]{UNAM.png}
           \end{flushright}
        \end{minipage}
        \begin{minipage}[H]{0.62 \textwidth}
            \begin{center}
                {\large \textsc{Universdad Nacional Autónoma de México}}
                \vspace{0.25cm}
                \\
                { \large \textbf{Lenguajes de Programacion\\ Examen Parcial I}}                
                \textbf{}
                \begin{multicols}{2}
                \begin{flushleft}
                \begin{itemize}
                    % NOMBRES DE INTEGRANTES
                    \item  \small Edgar Montiel Ledesma\\ 317317794
    
                    \item \footnotesize Carlos Daniel Cortes Jimenez\\ 420004846
                \end{itemize}
                \end{flushleft}
                \vspace{0.25cm}
                \end{multicols} 
            \end{center}
            \vspace{0.05cm}
        \end{minipage}
        \begin{minipage}[t]{0.16 \textwidth}
            \begin{flushleft}
                \includegraphics[width=1in]{EFC.png}
            \end{flushleft}
        \end{minipage}
    }
    
    \begin{tikzpicture}
        \draw[thick] (-6.5,0)--(11.2,0);
    \end{tikzpicture}
    %%%%%%%%%%%%%%%%%%%%%%%%%%%%%%%%%%%%%%%%%%%%%%%%%%%%%%%%%
    \begin{itemize}
        \item Chon Hacker quiere quiere desarrollar un lenguaje calculador para expresiones aritméticas en notación polaca. La notación polaca es un antiguo sistema que no requiere de paréntesis, para lograrlo mueve los operadores al final de la expresión, es decir, utiliza notación posfija.\\
        Por ejemplo, la expresión $1 + 2$ se convierte en $2 1 +$ y la expresión $1 - (3 + 2)$ corresponde a $1 3 2 + -$.\\
        Este lenguaje evalúa las expresiones metiendo cada símbolo a una pila hasta que se encuentra un operador, al ver un operador saca los dos símbolos que se encuentran en el tope de la pila, evalúa la operación con ellos y agrega el resultado a la pila.
        Una gramática con la que se puede definir este lenguaje es la siguiente:
        \begin{center}
            \begin{itemize}
                \item[ ] sym $::= n \,| + \,| - \,| \,*$
                \item[ ] rpn $::= \epsilon \,| \,sym \,rpn$
            \end{itemize}
        \end{center}
        Responde a los siguientes incisos:\\
        \begin{itemize}
            \item[a)] Con la gramática anterior se pueden construir programas que no pertenecen al lenguaje como $+1 2$ o $1 + 2$. Para tratar con esto Chon Hacker te pide definir mediante reglas de inferencia una semántica estática que verifique que la expresión es correcta según la definición de la notación polaca. Para esto se define el juicio e correct que indica que la expresión e está correctamente en notación polaca.
            \item[b)] Usando la definición del inciso anterior verifica que $1 2 4 3 + * -$ correct
            \item[c)] Define una semántica operacional de paso grande para este lenguaje calculador mediante la relación $\Downarrow$ que evalúa los programas del lenguaje. Puede ser de ayuda leer el programa de derecha a izquierda.
            \item[d)] Define una semántica operacional de paso pequeño con estados de la forma $s \models e$ en donde s es la pila que almacena los símbolos (la pila vacía se denota como $\circ)$ y e es la expresión del lenguaje. Para esto indica claramente cuáles son los estados iniciales y finales y define la relación de transición $\rightarrow$rpn que modela la evaluación del programa.
            \item[e)] Prueba que las semánticas definidas en los incisos anteriores son equivalentes, para esto:\\

            Demuestra que para cualquier expresión correcta e del lenguaje, si e $\Downarrow$ v entonces $\circ$ $\models$ e $\rightarrow$ $*$ rpn $\circ$ $\models$  v.
        \end{itemize}
    \end{itemize}
\end{document}
